%%%%%%%%%%%%%%%%%%%%%%%%%%%%%%%%%%%%%%%%%
% FRI Data Science_report LaTeX Template
% Version 1.0 (28/1/2020)
% 
% Jure Demšar (jure.demsar@fri.uni-lj.si)
%
% Based on MicromouseSymp article template by:
% Mathias Legrand (legrand.mathias@gmail.com) 
% With extensive modifications by:
% Antonio Valente (antonio.luis.valente@gmail.com)
%
% License:
% CC BY-NC-SA 3.0 (http://creativecommons.org/licenses/by-nc-sa/3.0/)
%
%%%%%%%%%%%%%%%%%%%%%%%%%%%%%%%%%%%%%%%%%


%----------------------------------------------------------------------------------------
%	PACKAGES AND OTHER DOCUMENT CONFIGURATIONS
%----------------------------------------------------------------------------------------
\documentclass[fleqn,moreauthors,10pt]{ds_report}
\usepackage[english]{babel}
\usepackage{newtxtext, newtxmath}

\graphicspath{{fig/}}




%----------------------------------------------------------------------------------------
%	ARTICLE INFORMATION
%----------------------------------------------------------------------------------------

% Header
\JournalInfo{FRI Natural language processing course 2025}

% Interim or final report
\Archive{Project report} 
%\Archive{Final report} 

% Article title
\PaperTitle{Conversational Agent with Retrieval-Augmented Generation} 

% Authors (student competitors) and their info
\Authors{Katarina Velkov, Nejc Krajšek, Luka Sabotič}

% Advisors
\affiliation{\textit{Advisors: Aleš Žagar}}

% Keywords
\Keywords{Conversational agent, Retrieval-Augmented Generation}
\newcommand{\keywordname}{Keywords}


%----------------------------------------------------------------------------------------
%	ABSTRACT
%----------------------------------------------------------------------------------------

\Abstract{

}

%----------------------------------------------------------------------------------------

\begin{document}

% Makes all text pages the same height
\flushbottom 

% Print the title and abstract box
\maketitle 

% Removes page numbering from the first page
\thispagestyle{empty} 

%----------------------------------------------------------------------------------------
%	ARTICLE CONTENTS
%----------------------------------------------------------------------------------------

\section*{Introduction}
Large language models (LLM) rely solely on their pre-trained knowledge to generate responses. These models have proved to be powerful but they are prone to generating hallucinated or inaccurate information. Furthermore their use in specialized fields has proved to be challenging as LLMs are trained mostly on publicly available data. That is why the need for more advanced models has emerged. Retrieval-Augmented Generation (RAG) enables LLMs to retrieve more relevant information from various internal databases and web sources during conversations. This process allows for more accurate, domain-specific and up-to-date responses.

The aim of this project is to build a conversational agent using the RAG technique. We will evaluate the performance of our model by comparing it to GPT-4 / BERT models.


%------------------------------------------------

\section*{Related work}

Implementation of Retrieval-Augmented Generation (RAG) was first introduced in 2020 and since then the amount of contributions has been growing. A. Afzal et al. (2024) reported optimization to enhance performance in the academic domain.

%------------------------------------------------

\section*{Methods}

%------------------------------------------------

\section*{Results}

%------------------------------------------------

\section*{Discussion}


%------------------------------------------------

\section*{Acknowledgments}

%----------------------------------------------------------------------------------------
%	REFERENCE LIST
%----------------------------------------------------------------------------------------
\bibliographystyle{unsrt}
\bibliography{report}


\end{document}